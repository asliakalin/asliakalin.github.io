\documentclass[11pt]{article}

\usepackage[margin=1in]{geometry}
\usepackage{fancyhdr}
\pagestyle{fancy}
\usepackage{amsmath}
\usepackage{amssymb}
\usepackage{multirow}
\usepackage{array}
\usepackage{tikz}
\usetikzlibrary{calc,trees,positioning,arrows,fit,shapes,calc}

\lhead{Proofs}
\chead{Asli Akalin | Note 2}
\rhead{Fall 2018}


\begin{document}


\newpage
\section*{Proof Methods}
\begin{tabular}{ |m{10em}|m{10em}|m{11em}|m{11em}| } 
 \hline
 Direct Proof & by Contraposition & Contradiction & by Cases \\
 \hline 
 \hline
 $ P \Rightarrow Q $ & $ \neg Q \Rightarrow \neg P $ & $ P $ & $P$ \\ 
 \hline
 
 
Assume P, tweak stuff therefore get Q. & 
Assume $\neg Q$, tweak stuff, therefore get $\neg P$ & 
Assume $\neg P$, tweak stuff, show contradiction by showing that $\neg P \Rightarrow R$ and $ \neg P \Rightarrow \neg R$. Since the individual implications has to be True, $ \neg P \Rightarrow (R \wedge \neg R)$ has to hold. Notice $(R \wedge \neg R)) \equiv F$. So $ \neg P \Rightarrow F$ can only hold if $\neg P$ is false, ($F \Rightarrow F \equiv T$). So, $\neg P \equiv F \Rightarrow P \equiv T$. Thus P has to be T. & 
Basically, if nothing else works, list all of exhaustive cases and see what happens. \\

\hline 

-When you can use the info given with P to derive info that eventually leads you to Q & 
- When the info given from P is vague or hard to derive info from, assuming $\neg Q$ gives you more information and easier to derive conclusions from & -When there is only a single statement to be proven without an implication & -Usually when nothing else works\newline -But technically you can do all proofs using by cases, if you are determined enough. \\

\hline
Can use when: & Can use when: & Can use when: & Can use when: \\
- specific definitions given in P such as perfect square, odd, divisible by 5 etc can be used as stepping stones to get to next step \newline
- In general, if assuming P immediately gives you information about other variables in the question \newline
- For example, for odd n, you can immediately say if $n = m^2$, then m has to be odd & 

- P doesn't give enough information to move forward, is to vague or hard to assume and $\neg Q$ is easier to assume, use and expand from \newline 
- For example if you are given P = $n > 6$, that statement covers all n bigger than 6, which is infinitely many n values, hard to build on since requires multiple values to satisfy everything you will show \newline 
- Also helps if $\neg Q$ is a simple statement that can easily be negated & 

- Mostly when there is a simple statement like "there exists," "there is" without an explicit implication  given \newline
- P is easy to negate and develop on: \newline
- For example we can negate "there are infinite prime numbers" as "there are finite prime numbers" which implies that "there is a largest prime number" \newline
- "at least 2 people have same number of friends" can be negated as "nobody has same number of friends" which would imply "everybody has a unique number of friends"  & 

- Either case a, or case b, or case c ... has to occur but it is not possible to know which one will, so the idea is to cover all of them. \newline
- Can only be used when all possible cases are known, so be careful with infinite sets. \newline
- For example, the values n can take in the set Natural Numbers can be separated into cases as n=0, n=1, n=2 ... (which is hard to take into account one by one) or as $n<4, n=4, n>4$ etc, so define cases carefully. \\ 
\hline
\end{tabular}


\newpage
\section*{Useful Notes}
\begin{enumerate}
\item Basic Negation Keypoints (notice $\rightarrow$ is not an implication here but shows the negation)
\begin{enumerate}
\item everybody does P $\rightarrow$  there exists (at least one) person that does not P
\item there exists (at least one) n such that P $\rightarrow$ all n are (not P)
\item $A > B$ $\rightarrow$ $A<= B$ 
\end{enumerate}

\item To have a proof of "if and only if" have to show the original implication and its converse to be true. If converse holds for an implication$P \Rightarrow Q$, then $ P \equiv Q$.
\item Proof by contraposition relies on the fact that contraposition of an implication is logically equivalent to the original implication.
\item Contraposition proof is essentially the direct proof of contrapositive, the only difference is that you start by assuming a the negation of RHS. So, if the goal is to prove an implication, decide between direct proof or contraposition by deciding whether P or $\neg Q$ is easier to assume.
\item Proof by contradiction especially works well with proving something does not exist. Start by assuming that it does exist and show why it would break the universe if it did exist. (Key idea: assume negation to show some nonsense/contradiction, but since all the steps were legit based on that assumption, the only explanation is that the assumption itself was wrong)



\end{enumerate}


\end{document}
