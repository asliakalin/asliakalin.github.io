\documentclass[11pt]{article}

\usepackage[margin=1in]{geometry}
\usepackage{fancyhdr}
\pagestyle{fancy}
\usepackage{amsmath}
\usepackage{amssymb}
\usepackage{tikz}

\lhead{Induction and More!}
\chead{Asli Akalin | Note 3}
\rhead{Summer 2018}


\begin{document}
\section*{What is Induction}
\begin{enumerate}
\item So far we've learned some proof techniques that builds on specific definitions, conditions or implications. We had to make some assumptions to show how one statement led to the other statement by using logical deductions. Now we want to show it holds
\item Induction is another proof technique we use in order to show that a statement holds for all natural numbers.
\item Principle of Induction, mathematically, says $ P(0) \wedge (\forall n \in N) (P(n) \Rightarrow P(n+1)) $. 
\item What induction definition essentially means is that if some statement P holds for P(0), P(1) must hold and if P(1) holds, then P(2) must hold, and if P(2) holds then P(3) must hold and so on. So to prove that the statement P(x) holds for $x \in N$, all we have to show is that P(0) holds and the implication $P(n) \Rightarrow P(n+1) $ holds.
\item This idea of stepping on the previous n value to get to n+1 is often can be thought as climbing a ladder. If we want to prove that we can climb to any step on a ladder, we have to prove two things:

\begin{tikzpicture}

\draw (0, -4.5) -- (0.5, -4) -- (0.5, -3.5) -- (1,-3.5) -- (1,-3) -- (1,-3) -- (0,-2.5) -- (0,-2.5) -- (1,-2) -- (1,-2) -- (2,-1.5) -- (2,-1.5);

\end{tikzpicture}

\begin{enumerate}
\item We can get on the ladder (if you can't get on the ladder, you clearly cannot get to any step of the ladder): $\Rightarrow P(0)$ must hold.
\item We are capable of taking a step once on the ladder (if you are not capable of taking a step, then clearly no matter where you are on the ladder, you won't be able to get to next step, you'll just be stuck at wherever you are, so you can't get to any step you want in the ladder): $P(n) \Rightarrow P(n+1)$ must hold.
\end{enumerate}
\begin{enumerate}
\item
\begin{enumerate}
\item s = \{1,-1\}
\item s = \{1,2,3,4,5,6,7,8,9,10,11\}
\item s = \{0,1,4,9,16,25,36,49,64,81\}
\item $\emptyset$
\end{enumerate}

\item b
\begin{enumerate}
\item i
\item ii
\item iii
\item iv
\end{enumerate}

\item c
\end{enumerate}

\item two
\item three
\item four


\end{enumerate}

\section*{Truth Tables}
\end{document}